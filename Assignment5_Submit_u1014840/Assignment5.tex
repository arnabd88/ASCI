\documentclass{article}
\usepackage{algorithm}
\usepackage{algpseudocode}
\usepackage{amsmath,amssymb,amsthm}
\usepackage{graphicx}
\usepackage[margin=1in]{geometry}
\usepackage{fancyhdr}
\usepackage{float}
\usepackage{longtable}
\newcommand{\bx}{{\bf x}}
\newcommand{\bw}{{\bf w}}
\newcommand{\bb}{{\bf b}}
\newcommand{\bv}{{\bf v}}
\newcommand{\by}{{\bf y}}
\setlength{\parindent}{0pt}
\setlength{\parskip}{5pt plus 1pt}
\setlength{\headheight}{13.6pt}
\newcommand\question[2]{\vspace{.25in}\hrule\textbf{#1: #2}\hrule\vspace{.10in}}
\renewcommand\part[1]{\vspace{.10in}\textbf{(#1)}}
\newcommand\algo{\vspace{.10in}\textbf{Algorithm: }}
\newcommand\correctness{\vspace{.10in}\textbf{Correctness: }}
\newcommand\runtime{\vspace{.10in}\textbf{Running time: }}
\newcommand\pseudoCode{\vspace{.10in}\textbf{PseudoCode: }}
\newcommand*{\perm}[2]{{}^{#1}\!P_{#2}}
\newcommand*{\comb}[2]{{}^{#1}\!C_{#2}}
%\pagestyle{fancyplain}
%\lhead{\textbf{\NAME\ (\UID)}}
%\chead{\textbf{Hw\HWNUM}}
%\rhead{CS 6350, \today}
\title{CS6210 - Homework/Assignment-5}
\author{Arnab Das(u1014840)}
\usepackage[utf8]{inputenc}
\begin{document}
  \pagenumbering{gobble}
  \maketitle
  \newpage
  \pagenumbering{arabic}
  \newcommand\NAME{ARNAB DAS}
  \newcommand\UID{uxxxxxxx}
  \newcommand\HWNUM{4}



\question{1}{Chapter-9: Question-5}
An $n \times n$ linear system of equations $Ax=b$ is modified such that every $b_i$ is replaced with $b_i - x_i^3$. \newline\part{a} For the system $Ax=b$, thus we have: \newline
A = $\begin{bmatrix}
	a_{11} & a_{12} & a_{13} & \dots & a_{1n} \\
	a_{21} & a_{22} & a_{23} & \dots & a_{2n} \\
	\dots & \dots & \dots & \dots & \dots \\
	a_{n1} & a_{n2} & a_{n3} & \dots & a_{nn} \\
\end{bmatrix}$ and $A_{2}$ = 



\question{7}{Chapter-10: Question-25}

\part{a} Let the quadratic polynomial be 
\begin{equation}
	p(x) = c_0 + c_1 x + c_2 x^2
	\label{eq:quad}
\end{equation}

Then , at x=0:
\[ p(0) = c_0 = f(0) = 1 \]
and
\[ p^\prime(0) = c_1 = f^\prime(0) = -1\]

At, x=1:
\[ c_0 + c_1 + c_2 = f(1) = 2 \]
Solving the above 3 equations, gives
\[ c_0 = 1\]
\[ c_1 = -1\]
\[ c_2 = 2\]

Plugging these coefficients back to $\eqref{eq:quad}$, we get the following quadratic interpolant:
\begin{equation}
	p(x) = 1 - x + 2x^2
	\label{eq:poly}
\end{equation}

By taking the first derivative and the second derivative of $\eqref{eq:poly}$, we get:
\[p^\prime(x) = -1 + 4x\]
\[p^{\prime\prime}(x) = 4 > 0\]
Equating the first derivative to 0, we get a stationary point at $x=\dfrac{1}{4}$, and since the second derivative, $p^{\prime\prime}(x) > 0$, hence this point is a minima. Thus we have found  $x^* = \dfrac{1}{4}$. (Answer) \newline

\part{b}
We already found the minima $x^* = \dfrac{1}{4}$ which is $0 < \dfrac{1}{4} < 1$. Since, it is quadratic, it has only a single maxima or minima becuase its first derivative is linear and hence cannot have multiple roots. Since, we have found a minima, hence it has to be the unique minima satisfying $0 < x^* < 1$. 
For the original function, given it has continuous second order derivative, however nothing is mentioned about its higher derivatives. Now, since $f(1) > f(0)$, there is a point in $[0,1)$, from where the function f starts to increase. Since, $f^\prime(0) < 0$, hence $x=0$ is not the point where f achieves a minimum. So, f achieves a minimum at $x^* = 0 + \delta$, where $\delta$ is a positive quantity between 0 and 1. Hence, f has a unique minimum at a point $0 < x < 1$ .Since , f has continuous second order derivative in this region, we do not know about its higher order derivatives. If it is quadratic, then this minima is unique for same reasons as the quadratic polynomial. But , if the underlying function is not quadratic, then the uniqueness cannot be asserted since there can be many more minimas in this region. 


\question{8}{Chapter-11: Question-3}
Let $f \in C^3[a,b]$ be given at equidistant points, $x_i = a + ih,$ where $i=0,1,2, \dots, n$ and $nh = b-a$. $f^\prime(a)$ is given as well. \newline

\part{a} Construct an algorithm for $C^1$ continuous quadratic interpolation: \newline
\[ v(x) = S_i(x) = a_i + b_i(x - x_i) + c_i(x - x_i)^2 ; x_i \leq x \leq x_{i+1}\]
That is an algorithm to determine the 3n coefficients. \newline

\algo
 The polynomial $S_i(x)$ in the interval $x_i \leq x \leq x_{i+1}$ will match the function value exactly. So, we have our first condition as:
	\[S_i(x_i) = a_i = f(x_i) \]
Thus, we have n equations in n variables due to this.
This gives us all the n $a_i$ values. \newline

Also, since $S_i(x)$ is valid in the interval $x_i \leq x \leq x_{i+1}$, hence it should satisfy the function value at $x_{i+1}$ as well. Thus we have our next condition as: 
\[ S_i(x_{i+1}) = f(x_{i+1}) \]
\[ a_i + b_i(x_{i+1} - x_i) + c_i(x_{i+1} - x_i)^2 = f(x_{i+1})\]
since, $x_{i+1} - x_i = h$, we get:
\[ b_i h + c_i h^2 = a_{i+1} - a_i\]
This gives us n equations in 2n variables. In combination with the above we get 2n equations in 3n variables. \newline

Next, since we want the interpolant to be $C^1$ continuous and the given fucntion is $C^3$, hence, the first derivative of $S_i(x_i)$ should match the first derivative of the function at $x_i$. Thus, we get: 
\[ S_{i}^\prime(x_i) = f^\prime(x_i) \]
\[ b_i = f^\prime(x_i)\]

Also, the derivatives should match at $x_{i+1}$. Hence, 
\[ S_i^\prime(x_{i+1}) = f^\prime(x_{i+1})\]
\[ b_i + 2c_i(x_{i+1} - x_i) = f^\prime(x_{i+1})\]
\[ b_i + 2c_i h = f^\prime(x_{i+1})\]

There is a certain pattern that emerges. If we know the derivative value at $x_i$, we can use it further to compute the derivative value at $x_{i+1}$. Since, we need to evaluate all the derivative values exceopt the first one, let us denote the derivative with the variable $d_i$. \newline
Thus, from the above relations we have the following set of equations:

n equations for
\[ a_i = f(x_i)\]

n equations for
\[ b_i h + ci h^2 - (a_{i+1} - a_i) = 0\]

n equations for 
\[ b_i - d_i = 0\]

(n-1) equations for 
\[ b_i + 2c_i h - d_{i+1} = 0 \]

1 equations for the derivative given at x=a, that is $d_0$
\[ d_0 = f(a)\]

Thus, in total we have 4n variables with 4n equations which can be solved to get the 3n coefficients $\{a_i, b_i, c_i\}$ and the derivatives $d_i$. \newline

\textbf{One more method:} Since, we want the piecewise interpolant to be $C^1$ continuous, hence at the interior point $x_i$, the first derivative of $S_{i-1}(x)$ should equal the first derivative of $S_i(x)$. Hence, we have the following relation:
\[ S_{i-1}(x_i) = S_i(x_i)\]
\[ b_{i-1} + 2c_{i-1}(x_i - x_{i-1}) = b_i\]
\[ b_{i-1} - b_i + 2c_{i-1} h = 0\]
This will have (n-1) equations since we equate between two consecutive interior points. \newline
Thus, now we can consider the following set of equations: \newline
n equations for
\[ a_i = f(x_i)\]

n equations for
\[ b_i h + ci h^2 - (a_{i+1} - a_i) = 0\]

(n-1) equations for 
\[ b_{i-1} - b_i + 2c_{i-1} h = 0\]

and 1 equation for the condition that the derivative at $x=a$, $f^\prime(a)$, is given.
\[S_0^\prime(a) = b_0 = f^\prime(a) \]
\[ b_0 - f^\prime(a) = 0\]

Thus, now we have 3n equations and 3n variables, which can be solved to get the 3n coefficients $\{a_i, b_i, c_i\}$. \newline

 \part{b} The error estimate for an n-degree interpolate, $p_n(x)$, with respect to the original function, $f(x)$, is expressed as: 
 \[ f(x) - p_n(x) = \dfrac{f^{(n+1)}(\zeta)}{(n+1)!} \prod_{i=0}^n (x - x_i)\]

 For piecewise quadratic interpolant , $v(x) = S_i(x)$, we have n=2. Hence, we get,
 \begin{equation}
  |f(x) - v(x)| \leq (x - x_{i-1})(x - x_i) \max_{a \leq \zeta \leq b} \dfrac{f^3(\zeta)}{3!} 
  \label{eq:err}
 \end{equation}
 That is the 3rd derivative is bounded by its max value at a point over the entire interval [a,b]. 
 For the term, $(x-x_{i-1})(x - x_i)$, the maximum occurs at the midpoint of the interval , that is, at $\dfrac{x_{i-1} + x_i}{2}$. Thus, we get: \newline
 \[ |(x - x_{i-1})(x - x_i)| \leq \bigg ( \dfrac{x_i - x_{i-1})}{2} \bigg )^2 \]
 and since, $x_i - x_{i-1} = h$, we get
 \[ |(x - x_{i-1})(x - x_i)| \leq \bigg ( \dfrac{h}{2} \bigg )^2 \]

 Plugging this back to $\eqref{eq:err}$, we get the following error bound: \newline
 \[ ErrorBound = |f(x) - v(x)| \leq \dfrac{h^2}{24} \max_{a \leq \zeta \leq b} |f^3(\zeta)|\]








 \question{9}{Chapter-11: Question-12}
Derive a B-spline basis representation for piecewise linear interpolation and for piecewise cubic interpolation: \newline

For (n+1) interpolating  points and degree (k-1), the bspline representation is written as: \newline
\begin{equation}
  P(u) = \sum_{i=0}^n pi N_{i,k}(u)
  \label{eq:bsum}
\end{equation}
 where $N_{i,k}$ are the bspline basis functions and (k-1) is the degree of the curve. Hence, for a piece-wise linear interpolant, the degree of the curve is 1, hence $k=2$. Thus, we have (n+1) interpolating points and k=2. \newline
The idea of bspline formulation us to  generate curve that are affected locally by the control points. So, if a point is changed, the curve formulation will change locally, and not cause changes in the entire curve unlike bezier curves. The entire interval of points $\{ p_0, p_1, \dots, p_n\}$ is divided into a sequence of knots $(t_0, t_1, \dots, t_r)$, such that the effect of a point will have its control in a knot region $[t_{i-1}, t_i]$ and not anywhere else: \newline
The basis function, $N_{i,k}$, is defined as :\newline

\begin{equation}
   N_{i,k}(u) = \dfrac{(u - t_i) N_{i,k-1}(u)}{t_{i+k} - t_i} + \dfrac{(t_{i+k} - u)N_{i+1,k-1}(u)}{t_{i+k} - t_{i+1}}
  \label{eq:basis}
\end{equation}

The base case of the basis functions is given as:
\[ N_{i,1}(u) = 1; t_i \leq u < t_{i+1}\]
\[ = 0 ; otherwise\]

This allows for the termination condition of the recursion defined in $\eqref{eq:basis}$: \newline

In general, the knot values, $t_i$ are defined as: \newline
\[ t_i = 0; i < k\]
\[ t_i = i-k+1; k \leq i \leq n\]
\[ t_i = n-k+2; i > n\]

For the piecewise linear case, the value of $k=2$, hence the above relations translate to: 
\[ t_i = 0; i < 2\]
\[ t_i = i-1; 2 \leq i \leq n\]
\[ t_i = n; i > n\]

Thus , the set of $(n+k+1) = (n+3)$ knot values will be: \newline
\[ t = (0,0,1,2, \dots, n-1, n,n)\]

$\eqref{eq:bsum}$ in this case translates to : \newline
\[ P(u) = \sum_{i=0}^n p_i N_{i,2}(u) = p_0 N_{0,2}(u) + p_1 N_{1,2}(u) + \dots + p_i N_{i,2}(u) + \dots + p_n N_{n,2}(u) \]

where, the basis function $N_{i,2}$ in this case will be : \newline
\begin{equation}
   N_{i,2}(u) = \dfrac{(u - t_i) N_{i,1}(u)}{t_{i+2} - t_i} + \dfrac{(t_{i+2} - u)N_{i+1,1}(u)}{t_{i+2} - t_{i+1}}
   \label{eq:linbasis}
\end{equation}
Here, the base cases of the recursion will become: \newline
$N_{0,1} = 1 $ if $t_0 \leq u < t_1 $ or $0 \leq u < 0$ \newline
$N_{0,1} = 0, otherwise$, \newline
$N_{0,1} = 0$ \newline \newline
$N_{1,1} = 1$ if $t_1 \leq u < t2$ or $0 \leq u < 1$ \newline
$N_{1,1} = 0$, otherwise \newline \newline
$N_{2,1} = 1$ if $t_2 \leq u < t_3$ or $1 \leq u < 2$ \newline
$N_{2,1} = 0$, otherwise \newline \newline
$\dots$ \newline
$N_{n,1} = 1$ if $t_n \leq u < t_{n+1}$ or $n-1 \leq u < n$  \newline
$N_{n,1} = 0$, otherwise \newline \newline

The above base cases plugged into the recursion of the $eqref{eq:linbasis}$ gives the representation for the bspline basis function for piecewise linear. (Answer). \newline

For the case of the piecewise cubic hermiote interpolant, we have a degree 3 interpolant. Hence, (k-1) is 3, thus k=4. Hence, the relation for deriving the knot values will become: \newline
\[ t_i = 0; i < 4\]
\[ t_i = i-3; 4 \leq i \leq n \]
\[ t_i = n-2; i > n\]

Thus, the set of $(n+k+1) = (n+5)$ knot values will be: \newline
\[ t = (0,0,0,0,1,2, \dots, n-3, n-2,n-2,n-2,n-2)\]

$\eqref{eq:bsum}$ in this case translates to : \newline
\begin{equation}
 P(u) = \sum_{i=0}^n p_0 N_{0,4}(u) + p_1 N_{1,4}(u) + \dots + p_i N_{i,4}(u) + \dots + p_n N_{n,4}(u) 
 \label{eq:cubasis}
\end{equation}

Where the recurssions over the basis functions can be further expanded following the relation(1), to reach the terminal base cases which in these case will be defined as follows: \newline
$N_{0,1} = 1 $ if $t_0 \leq u < t_1$ or $0 \leq u < 0$ \newline
$N_{0,1} = 0$, otherwise, \newline
So, $N_{0,1} = 0$ \newline \newline

$N_{1,1} = 1 $ if $t_1 \leq u < t_2$ or $0 \leq u < 0$ \newline
$N_{1,1} = 0$, otherwise, \newline
So, $N_{1,1} = 0$ \newline \newline

$N_{2,1} = 1 $ if $t_2 \leq u < t_3$ or $0 \leq u < 0$ \newline
$N_{2,1} = 0$, otherwise, \newline
So, $N_{2,1} = 0$ \newline \newline

$N_{3,1} = 1$ if $t_3 \leq u < t_4$ or $0 \leq u < 1$ \newline
$N_{3,1} = 0$, otherwise, \newline \newline

$N_{4,1} = 1$ if $t_4 \leq u < t_5$ or $1 \leq u < 2$ \newline
$N_{4,1} = 0$, otherwise, \newline \newline

$N_{5,1} = 1$ if $t_5 \leq u < t_6$ or $2 \leq u < 3$ \newline
$N_{5,1} = 0$, otherwise, \newline 

$\dots$ \newline

$N_{n,1} = 1$ if $t_n \leq u < t_{n+1}$ or $(n-3) \leq u < (n-2)$ \newline
$N_{n,1} = 0$, otherwise, \newline \newline

Now consider $\eqref{eq:cubasis}$, the basis functions recursively depends exactly on 4 base cases if we break them down as follows: \newline
\[ N_{0,4} : (N_{0,1}, N_{1,1}, N_{2,1}, N_{3,1})\]
\[ N_{1,4} : ( N_{1,1}, N_{2,1}, N_{3,1}, N_{4,1})\]
\[ N_{2,4} : ( N_{2,1}, N_{3,1}, N_{4,1}, N_{5,1})\]
\[ N_{3,4} : ( N_{3,1}, N_{4,1}, N_{5,1}, N_{6,1})\]

and so on, \newline

In the region , $0 \leq u < 1$, only $N_{3,1}=1$, other $N_{i,1}'s$ are zero. Now , $N_{3,1}$ is present only in the recursive list of the basis functions associated with  $(p_0, p_1, p_2, P_3)$. Hence, in the region $0 \leq u < 1$, only these 4 points serve as the control points with local control over the curve. Other points do not have any effect here. Similarly, in the region $1 \leq u < 2$, only $N_{4,1}$ is 1, which is present in the recursive list of the basis functions associated with $(p_1, p_2, p_3,p_4)$. Hence, in the region $1 \leq u < 2$, only these 4 points serve as the local control points. \newline

Thus plugging in the base cases , $N_{i,1}'s$, present in the recursive list of respective basis functions, we obtain the representation of the bspline basis functions. (Answer). \newline











\end{document}
