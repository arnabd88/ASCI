\documentclass{article}
\usepackage{algorithm}
\usepackage{algpseudocode}
\usepackage{amsmath,amssymb,amsthm}
\usepackage{graphicx}
\usepackage[margin=1in]{geometry}
\usepackage{fancyhdr}
\usepackage{float}
\usepackage{longtable}
\newcommand{\bx}{{\bf x}}
\newcommand{\bw}{{\bf w}}
\newcommand{\bb}{{\bf b}}
\newcommand{\bv}{{\bf v}}
\newcommand{\by}{{\bf y}}
\setlength{\parindent}{0pt}
\setlength{\parskip}{5pt plus 1pt}
\setlength{\headheight}{13.6pt}
\newcommand\question[2]{\vspace{.25in}\hrule\textbf{#1: #2}\hrule\vspace{.10in}}
\renewcommand\part[1]{\vspace{.10in}\textbf{(#1)}}
\newcommand\algo{\vspace{.10in}\textbf{Algorithm: }}
\newcommand\correctness{\vspace{.10in}\textbf{Correctness: }}
\newcommand\runtime{\vspace{.10in}\textbf{Running time: }}
\newcommand\pseudoCode{\vspace{.10in}\textbf{PseudoCode: }}
\newcommand*{\perm}[2]{{}^{#1}\!P_{#2}}
\newcommand*{\comb}[2]{{}^{#1}\!C_{#2}}
%\pagestyle{fancyplain}
%\lhead{\textbf{\NAME\ (\UID)}}
%\chead{\textbf{Hw\HWNUM}}
%\rhead{CS 6350, \today}
\title{CS6210 - Homework/Assignment-5}
\author{Arnab Das(u1014840)}
\usepackage[utf8]{inputenc}
\begin{document}
  \pagenumbering{gobble}
  \maketitle
  \newpage
  \pagenumbering{arabic}
  \newcommand\NAME{ARNAB DAS}
  \newcommand\UID{uxxxxxxx}
  \newcommand\HWNUM{4}


 \question{9}{Chapter-11: Question-12}
Derive a B-spline basis representation for piecewise linear interpolation and for piecewise cubic interpolation: \newline

For (n+1) interpolating  points and degree (k-1), the bspline representation is written as: \newline
\begin{equation}
  P(u) = \sum_{i=0}^n pi N_{i,k}(u)
\end{equation}
 where $N_{i,k}$ are the bspline basis functions and (k-1) is the degree of the curve. Hence, for a piece-wise linear interpolant, the degree of the curve is 1, hence $k=2$. Thus, we have (n+1) interpolating points and k=2. \newline
The idea of bspline formulation us to  generate curve that are affected locally by the control points. So, if a point is changed, the curve formulation will change locally, and not cause changes in the entire curve unlike bezier curves. The entire interval of points $\{ p_0, p_1, \dots, p_n\}$ is divided into a sequence of knots $(t_0, t_1, \dots, t_r)$, such that the effect of a point will have its control in a knot region $[t_{i-1}, t_i]$ and not anywhere else: \newline
The basis function, $N_{i,k}$, is defined as :\newline

\begin{equation}
   N_{i,k}(u) = \dfrac{(u - t_i) N_{i,k-1}(u)}{t_{i+k} - t_i} + \dfrac{(t_{i+k} - u)N_{i+1,k-1}(u)}{t_{i+k} - t_{i+1}}
\end{equation}

The base case of the basis functions is given as:
\[ N_{i,1}(u) = 1; t_i \leq u < t_{i+1}\]
\[ = 0 ; otherwise\]

This allows for the termination condition of the recursion defined in equation(2): \newline

In general, the knot values, $t_i$ are defined as: \newline
\[ t_i = 0; i < k\]
\[ t_i = i-k+1; k \leq i \leq n\]
\[ t_i = n-k+2; i > n\]

For the piecewise linear case, the value of $k=2$, hence the above relations translate to: 
\[ t_i = 0; i < 2\]
\[ t_i = i-1; 2 \leq i \leq n\]
\[ t_i = n; i > n\]

Thus , the set of $(n+k+1) = (n+3)$ knot values will be: \newline
\[ t = (0,0,1,2, \dots, n-1, n,n)\]

Equation(1) in this case translates to : \newline
\[ P(u) = \sum_{i=0}^n p_i N_{i,2}(u) = p_0 N_{0,2}(u) + p_1 N_{1,2}(u) + \dots + p_i N_{i,2}(u) + \dots + p_n N_{n,2}(u) \]

where, the basis function $N_{i,2}$ in this case will be : \newline
\begin{equation}
   N_{i,2}(u) = \dfrac{(u - t_i) N_{i,1}(u)}{t_{i+2} - t_i} + \dfrac{(t_{i+2} - u)N_{i+1,1}(u)}{t_{i+2} - t_{i+1}}
\end{equation}
Here, the base cases of the recursion will become: \newline
$N_{0,1} = 1 $ if $t_0 \leq u < t_1 $ or $0 \leq u < 0$ \newline
$N_{0,1} = 0, otherwise$, \newline
$N_{0,1} = 0$ \newline \newline
$N_{1,1} = 1$ if $t_1 \leq u < t2$ or $0 \leq u < 1$ \newline
$N_{1,1} = 0$, otherwise \newline \newline
$N_{2,1} = 1$ if $t_2 \leq u < t_3$ or $1 \leq u < 2$ \newline
$N_{2,1} = 0$, otherwise \newline \newline
$\dots$ \newline
$N_{n,1} = 1$ if $t_n \leq u < t_{n+1}$ or $n-1 \leq u < n$  \newline
$N_{n,1} = 0$, otherwise \newline \newline

The above base cases plugged into the recursion of the equation(3) gives the representation for the bspline basis function for piecewise linear. (Answer). \newline

For the case of the piecewise cubic hermiote interpolant, we have a degree 3 interpolant. Hence, (k-1) is 3, thus k=4. Hence, the relation for deriving the knot values will become: \newline
\[ t_i = 0; i < 4\]
\[ t_i = i-3; 4 \leq i \leq n \]
\[ t_i = n-2; i > n\]

Thus, the set of $(n+k+1) = (n+5)$ knot values will be: \newline
\[ t = (0,0,0,0,1,2, \dots, n-3, n-2,n-2,n-2,n-2)\]

Equation(1) in this case translates to : \newline
\begin{equation}
 P(u) = \sum_{i=0}^n p_0 N_{0,4}(u) + p_1 N_{1,4}(u) + \dots + p_i N_{i,4}(u) + \dots + p_n N_{n,4}(u) 
\end{equation}

Where the recurssions over the basis functions can be further expanded following the relation(1), to reach the terminal base cases which in these case will be defined as follows: \newline
$N_{0,1} = 1 $ if $t_0 \leq u < t_1$ or $0 \leq u < 0$ \newline
$N_{0,1} = 0$, otherwise, \newline
So, $N_{0,1} = 0$ \newline \newline

$N_{1,1} = 1 $ if $t_1 \leq u < t_2$ or $0 \leq u < 0$ \newline
$N_{1,1} = 0$, otherwise, \newline
So, $N_{1,1} = 0$ \newline \newline

$N_{2,1} = 1 $ if $t_2 \leq u < t_3$ or $0 \leq u < 0$ \newline
$N_{2,1} = 0$, otherwise, \newline
So, $N_{2,1} = 0$ \newline \newline

$N_{3,1} = 1$ if $t_3 \leq u < t_4$ or $0 \leq u < 1$ \newline
$N_{3,1} = 0$, otherwise, \newline \newline

$N_{4,1} = 1$ if $t_4 \leq u < t_5$ or $1 \leq u < 2$ \newline
$N_{4,1} = 0$, otherwise, \newline \newline

$N_{5,1} = 1$ if $t_5 \leq u < t_6$ or $2 \leq u < 3$ \newline
$N_{5,1} = 0$, otherwise, \newline 

$\dots$ \newline

$N_{n,1} = 1$ if $t_n \leq u < t_{n+1}$ or $(n-3) \leq u < (n-2)$ \newline
$N_{n,1} = 0$, otherwise, \newline \newline

Now consider equation(4), the basis functions recursively depends exactly on 4 base cases if we break them down as follows: \newline
\[ N_{0,4} : (N_{0,1}, N_{1,1}, N_{2,1}, N_{3,1})\]
\[ N_{1,4} : ( N_{1,1}, N_{2,1}, N_{3,1}, N_{4,1})\]
\[ N_{2,4} : ( N_{2,1}, N_{3,1}, N_{4,1}, N_{5,1})\]
\[ N_{3,4} : ( N_{3,1}, N_{4,1}, N_{5,1}, N_{6,1})\]

and so on, \newline

In the region , $0 \leq u < 1$, only $N_{3,1}=1$, other $N_{i,1}'s$ are zero. Now , $N_{3,1}$ is present only in the recursive list of the basis functions associated with  $(p_0, p_1, p_2, P_3)$. Hence, in the region $0 \leq u < 1$, only these 4 points serve as the control points with local control over the curve. Other points do not have any effect here. Similarly, in the region $1 \leq u < 2$, only $N_{4,1}$ is 1, which is present in the recursive list of the basis functions associated with $(p_1, p_2, p_3,p_4)$. Hence, in the region $1 \leq u < 2$, only these 4 points serve as the local control points. \newline

Thus plugging in the base cases , $N_{i,1}'s$, present in the recursive list of respective basis functions, we obtain the representation of the bspline basis functions. (Answer). \newline










\end{document}